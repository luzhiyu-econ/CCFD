\documentclass{beamer}
\usepackage{ctex, hyperref}
\usepackage[T1]{fontenc}

% other packages
\usepackage{latexsym,amsmath,xcolor,multicol,booktabs,calligra}
\usepackage{graphicx,pstricks,listings,stackengine}

\title[Geoeconomics and Power]{Geoeconomics: Understanding the Power Dynamics in International Relations}
\subtitle{A Theoretical Framework for Analyzing Economic Coercion}
\institute[中央财经大学]{中国财政协同发展创新中心}
\date{\today}

\usepackage{cufe}

% defs
\def\cmd#1{\texttt{\color{red}\footnotesize $\backslash$#1}}
\def\env#1{\texttt{\color{blue}\footnotesize #1}}
\definecolor{deepblue}{rgb}{0,0,0.5}
\definecolor{deepred}{rgb}{0.6,0,0}
\definecolor{deepgreen}{rgb}{0,0.5,0}
\definecolor{halfgray}{gray}{0.55}

\lstset{
    basicstyle=\ttfamily\small,
    keywordstyle=\bfseries\color{deepblue},
    emphstyle=\ttfamily\color{deepred},    % Custom highlighting style
    stringstyle=\color{deepgreen},
    numbers=left,
    numberstyle=\small\color{halfgray},
    rulesepcolor=\color{red!20!green!20!blue!20},
    frame=shadowbox,
}




\begin{document}
	
	\begin{frame}
		\titlepage
	\end{frame}
	
	\begin{frame}
		\frametitle{Table of Contents}
		\tableofcontents
	\end{frame}
	

\section{Introduction}

\begin{frame}
	\frametitle{Research Background and Importance}
	\begin{itemize}
		\item Geoeconomics is the use of a nation's fiscal and trade relations to achieve geopolitical and economic objectives.
		\item In the context of current competition between the US and China, the article aims to provide a model to conceptualize how major powers use their financial and economic strength to extract economic and political surpluses from countries around the world.
		\item Geoeconomic power is a form of soft power, operating through commercial channels such as disrupting the supply of goods or purchases, technology sharing, or financial relationships and services.
	\end{itemize}
\end{frame}

\section{Research Questions and Hypotheses}

\begin{frame}
	\frametitle{Research Questions and Hypotheses}
	\begin{itemize}
		\item How is geoeconomic power generated, and how can the exercise of threats across multiple economic activities increase the power in equilibrium?
		\item Hypotheses include input-output linkages, limited contract enforceability, and externalities.
		\item Geoeconomic power arises from the ability to exercise threats from independent economic activities in a coordinated manner.
	\end{itemize}
\end{frame}

\section{Model}

\begin{frame}
	\frametitle{The Model}
	\begin{itemize}
		\item The framework is based on three core elements: input-output linkages, limited contract enforceability, and externalities.
		\item The model includes production and consumer externalities, as well as limitations on contract enforceability.
		\item Geoeconomic power comes from a country's ability to integrate threats across multiple economic relationships, often with some of the threats being executed by third-party entities that are also under pressure.
	\end{itemize}
\end{frame}

\section{Derivation Process and Formula Interpretation}

\begin{frame}
	\frametitle{Derivation Process and Formula Interpretation}
	\begin{itemize}
		\item The article includes several mathematical derivations and formulas to explain different concepts and relationships in the model.
		\item For example, the Leontief inverse matrix is used to summarize the propagation of production externalities in the input-output network.
		\item Additionally, the article derives how a hegemonic country can exercise power by demanding its entities in the network to take costly actions, which can be markups on goods or increases in loan interest rates, or import restrictions and tariffs.
		\end{itemize}
\end{frame}

\section{Main Results}

\begin{frame}
	\frametitle{Main Results}
	\begin{itemize}
		\item The main result is that a hegemonic country can amplify its influence by controlling some strategic sectors, such as financial services, as the distribution changes in these sectors have a greater impact on the world economy.
		\item The article also explores how a hegemonic country can compete with other countries in the geoeconomic domain and how to provide the greatest combined threat in competition.
	\end{itemize}
\end{frame}

\section{Practical Significance}

\begin{frame}
	\frametitle{Practical Significance}
	\begin{itemize}
		\item The article provides a theoretical framework for analyzing and understanding how countries use their economic strength to achieve geopolitical objectives.
		\item This is significant for understanding issues such as economic sanctions, trade wars, and technology competition in current international relations.
	\end{itemize}
\end{frame}

\section{Conclusion}

\begin{frame}
	\frametitle{Conclusion}
	\begin{itemize}
		\item The article formalizes the concept of economic coercion by constructing a theoretical model that views it as a combination of strategic pressure and costly actions.
		\item By applying this framework, the author analyzes real-world cases such as national security externalities and China's Belt and Road Initiative, demonstrating the potential application of the model to real-world issues.
	\end{itemize}
\end{frame}

\section*{References}

\begin{frame}
	\bibliographystyle{plain}
	\bibliography{references}
\end{frame}

\end{document}
