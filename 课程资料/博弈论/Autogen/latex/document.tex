\documentclass{beamer}

\usetheme{Madrid}
\usecolortheme{beaver}

\title[Quack Markets]{Quack Markets: Limited Rationality and Market Dynamics}
\author[Research Agent]{Research Agent}
\date{\today}

\begin{document}
	
	\begin{frame}
		\titlepage
	\end{frame}
	
	\section{Introduction}
	
	\begin{frame}
		\frametitle{Research Background and Importance}
		\begin{itemize}
			\item A hypothetical market with quacks and patients competing on prices.
			\item Quacks provide treatments with no actual help for recovery.
			\item Recovery probability \( \alpha \) is fixed, regardless of treatment.
			\item Market should not exist if patients are rational.
			\item Patients' decisions are influenced by limited rationality and anecdotal reasoning.
			\item Limited rationality significantly impacts the market, leading to active markets, welfare losses for patients, and a non-monotonic relationship between market participants \( n \) and recovery probability \( \alpha \).
		\end{itemize}
	\end{frame}
	
	\section{Research Questions and Hypotheses}
	
	\begin{frame}
		\frametitle{Research Questions and Hypotheses}
		\begin{itemize}
			\item Can market competition mitigate the negative impact of quacks on patient welfare?
			\item A theoretical model is constructed assuming quacks are profit maximizers and patients follow limited rationality decision rules.
			\item Patients rely on random, anecdotal stories about treatment quality as complete information.
		\end{itemize}
	\end{frame}
	
	\section{Model Derivation and Formula Interpretation}
	
	\begin{frame}
		\frametitle{Model Derivation and Formula Interpretation}
		\begin{itemize}
			\item Patients make choices based on the S(1) procedure.
			\item For each alternative \( i \), patients sample once and get result \( x_i \), where \( x_i = 1 \) for success and \( x_i = 0 \) for failure.
			\item Patients choose alternative \( i \) that maximizes \( x_i - p_i \) in their sample.
			\item Quacks consider patient choice procedures when determining pricing strategies.
			\item The Nash equilibrium is unique, symmetric, and mixed-strategy.
			\item Quacks act as "cheaters," charging positive prices for worthless treatments.
			\item There is a negative correlation between \( \alpha \) and expected prices.
		\end{itemize}
	\end{frame}
	
	\section{Main Results}
	
	\begin{frame}
		\frametitle{Main Results}
		\begin{enumerate}
			\item Quack markets are active, with positive prices for worthless treatments.
			\item Patient welfare losses are non-monotonic in relation to \( n \) and \( \alpha \).
			\item In an extended model, quacks minimize price competition pressure by providing treatments with the greatest differentiation.
			\item Patient welfare losses are robust to market interventions that would eliminate low-quality firms in standard models but may be ineffective here.
			\end{enumerate}
	\end{frame}
	
	\section{Practical Significance}
	
	\begin{frame}
		\frametitle{Practical Significance}
		\begin{itemize}
			\item Understanding market interactions in "soft skill" industries like psychotherapy, management consulting, forecasting, and alternative medicine.
			\item Consumers often rely on anecdotes when facing unexpected issues and entering the market without a long learning phase.
			\item Examples of anecdotal reasoning: "A friend took this medicine and feels much better now," or "We should trust this political analyst because he foresaw the collapse of the Soviet Union."
		\end{itemize}
	\end{frame}
	
	\section{Conclusion}
	
	\begin{frame}
		\frametitle{Conclusion}
		\begin{itemize}
			\item Theoretical models show non-intuitive phenomena in market interactions between limited rationality consumers and rational firms.
			\item Markets can be active even with completely worthless services, and traditional competition policies may not improve consumer welfare.
			\item Findings are significant for understanding and designing more effective market interventions.
		\end{itemize}
	\end{frame}
	
	\section*{References}
	
	\begin{frame}
		\bibliographystyle{plain}
		\bibliography{references}
	\end{frame}
	
\end{document}
